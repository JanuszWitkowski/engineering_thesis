\chapter{Warcaby (wariant angielski)}
\thispagestyle{chapterBeginStyle}
\label{rozdzial1}

{\color{dgray}
Niniejszy rozdział poświęcony będzie zaznajomieniu Czytelnika z grą w Warcaby oraz szczególną wersją tej gry. Omówione również zostaną powody skupienia się nad tym wariantem, jak i inne badania przeprowadzone w przeszłości przez naukowców.
}

Wariant angielski Warcabów wprowadza tylko dwie zmiany do zasad gry. Po pierwsze, zwykłe piony nie mogą bić do tyłu. Po drugie, damkom ogranicza się możliwość poruszania się o dowolną liczbę pól do jednego sąsiedniego pola bądź jednego bijącego skoku na pole sąsiadujące ze zbijanym pionkiem, lecz wciąż mogą poruszać się we wszystkich kierunkach na ukos. Damki w tym wariancie można określić jako pionki które mogą poruszać się do tyłu.

Praca ta skoncentrowana jest na wariancie angielskim głównie z powodu ograniczenia na wielkość przestrzeni możliwych ruchów. Dzięki temu algorytm grający ma mniej przypadków do rozpatrzenia, co ułatwia wykonywanie badań i eksperymentów. Zastosowanie tego modelu gry stanowi również pewien punkt odniesienia - według artykułu kanadyjskiego zespołu Jonathan'a Schaeffer'a opublikowanego w 2007 roku, wariant angielski Warcabów został rozwiązany. \ldots



