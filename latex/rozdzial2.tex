\chapter{Idea rozwiązania i algorytmy}
\thispagestyle{chapterBeginStyle}
\label{rozdzial2}

% TODO: Wrzucić parę linków do bibliografii.

% {\color{dgray}
% W tym rozdziale przedstawione zostanie podejście autora do problemu opisanego w rozdziale~\ref{rozdzial1}. Każdy omawiany aspekt podejścia i każda idea wykorzystywanych algorytmów podparta zostanie pseudokodem bądź ilustracją, jeżeli zajdzie taka potrzeba.
% }

Głównym celem pracy jest stworzenie prostego modelu sztucznej inteligencji do grania w Warcaby w wariancie angielskim z pewną strategią. Istnieją różne podejścia do tego zagadnienia, spośród których najpopularniejszymi są te stosujące sieci neuronowe (zestawem danych byłaby np. baza meczów rozegranych na mistrzostwach na przestrzeni kilkunastu lat). Praca skupia się na bardziej nadzorowanej metodzie. Zastosowany został algorytm Minimax połączony z funkcją oceny heurystycznej. Do częściowego wyznaczenia funkcji oceny wykorzystano odmianę algorytmu genetycznego.

\section{Minimax}

Minimax jest szczególną wersją algorytmu przeszukującego w grafie. Jego idea jest bardzo prosta i zbliżona do ludzkiego rozumowania. Mając dany stan planszy oraz głębokość przeszukiwania, algorytm rekurencyjnie rozpatruje kolejne stany planszy symulując wykonanie jednego możliwego ruchu. Kiedy już osiągnie maksymalną głębokość przeszukiwań, funkcją oceny heurystycznej przypisuje wartości stanom, po czym zwraca tę wartość do swojego stanu-rodzica. Mając wartości oceny od każdego swojego dziecka, stan wybiera jedną z nich i przekazuje ją do swojego rodzica. Gdy wybór dojdzie do stanu będącego korzeniem drzewa przeszukiwań, algorytm wybierze jedną z dostępnych mu ocen i zwróci ruch do stanu któremu ta ocena odpowiada.

{\small
\begin{pseudokod}[H]
%\SetAlTitleFnt{small}
\SetArgSty{normalfont}
\SetKwFunction{Evaluate}{evaluateState}
\SetKwFunction{Children}{getChildren}
\SetKwFunction{Minimax}{minimax}
\KwIn{Stan gry $state$, flaga gracza $maximizing$, głębokość przeszukiwań $h$, rozpatrujący gracz $player$}
\KwOut{Wartość funkcji oceny $eval$}
\eIf{$h = 0$}{
$eval \leftarrow \Evaluate{state, player}$\;
}{
\eIf{$maximizing = True$}{
$maxEval \leftarrow -\infty$\;
\ForEach{$child \in \Children{state}$}{
$childEval \leftarrow \Minimax{child, False, h - 1, player}$\;
\If{$childEval \ge maxEval$}{
$maxEval \leftarrow childEval$
% $bestState \leftarrow child$
}
}
% $state \leftarrow bestState$
$eval \leftarrow maxEval$
}{
$minEval \leftarrow +\infty$\;
\ForEach{$child \in \Children{state}$}{
$childEval \leftarrow \Minimax{child, True, h - 1, player}$\;
\If{$childEval \le minEval$}{
$minEval \leftarrow childEval$
% $bestState \leftarrow child$
}
}
% $state \leftarrow bestState$
$eval \leftarrow minEval$
}
}
\caption{Prosty algorytm Minimax}\label{alg:mine}
\end{pseudokod}
}

Swoją nazwę algorytm zawdzięcza naprzemiennemu rozpatrywaniu ocen heurystycznych w drzewie przeszukiwań. W momencie gdy dany gracz wywołuje procedurę Minimaxa rozpatrując możliwe do wykonania przez niego ruchy, oznaczany jest jako gracz MAX. W powstałych stanach gry gracz symuluje tok rozumowania jego przeciwnika, rozpatrując ruchy za niego i oznaczając go jako gracza MIN. Stany na kolejnym poziomie w drzewie przeszukiwań rozpatruje gracz MAX, i tak dalej. Gdy gracz MAX dokonuje wyboru oceny do przekazania do stanu-rodzica, wybiera maksymalną wartość. Analogicznie, gracz MIN wybiera ocenę o minimalnej wartości.

\ldots PRZYKŁAD DZIAŁANIA MINIMAXA \ldots

\subsection{Alpha-Beta-prunning}

\ldots
Cięcia alfa-beta to rezygnacja z rozpatrywania innych podgałęzi ze względu na brak możliwości poprawienia wyniku. \ldots

\ldots PRZYKŁAD ALFA-BETA-CIĘĆ \ldots

\ldots

\section{Funkcja oceny heurystycznej}

Ze względu na ogromny rozmiar przestrzeni stanów w Warcabach, obecne komputery nie potrafią przeszukać jej całości w ,,rozsądnym'' czasie. Jeśli jednym z celów budowania sztucznej inteligencji są w miarę szybkie decyzje prowadzące do zwycięstwa w grze, należy ukrócić przeszukiwanie przestrzeni stanów i w jakiś sposób obejść się z niedoskonałą informacją posiadaną przez komputer. Ocena danego stanu na planszy ma wspomóc taki komputer w wyrobieniu ,,intuicji'' poprzez analizę sytuacji na planszy.

\FloatBarrier

Podejście w pracy do funkcji oceny heurystycznej polega na rozpatrzeniu wielu parametrów na planszy (np. liczba pionów, liczba damek przeciwnika, liczba ruchów), przemnożeniu wartości tych parametrów przez ustalone z góry wagi, a na koniec zsumowaniem powstałych iloczynów. Suma tych iloczynów to wartość funkcji oceny, którą przypisuje się pod dany stan gry.

{
% \small
\begin{center}
\begin{table}
\centering
{\footnotesize
\begin{tabular}{|c | c || c | c|}
 \hline
 Nr & Parametr & Nr & Parametr \\ %[0.5ex] 
 \hline\hline
 1 & Liczba sojuszniczych pionów & 31 & Liczba przeciwnych damek w środkowych wierszach \\ 
 \hline
 2 & Liczba sojuszniczych damek & 32 & . \\
 \hline
 3 & Liczba przeciwnych pionów & 33 & . \\
 \hline
 4 & Liczba przeciwnych damek & 34 & . \\
 \hline
 5 & Liczba sojuszniczych pionów przy ścianie & 35 & . \\
 \hline
 6 & Liczba sojuszniczych damek przy ścianie & 36 & . \\ 
 \hline
 7 & Liczba przeciwnych pionów przy ścianie & 37 & . \\
 \hline
 8 & Liczba przeciwnych damek przy ścianie & 38 & . \\
 \hline
 9 & Liczba ruchomych pionów gracza & 39 & . \\
 \hline
 10 & Liczba ruchomych damek gracza & 40 & . \\
 \hline
 11 & Liczba ruchomych pionów przeciwnika & 41 & . \\ 
 \hline
 12 & Liczba ruchomych damek przeciwnika & 42 & . \\
 \hline
 13 & Liczba możliwych ruchów gracza & 43 & . \\
 \hline
 14 & Liczba możliwych ruchów przeciwnika & 44 & . \\
 \hline
 15 & Istnienie bijącego ruchu gracza & 45 & . \\
 \hline
 16 & Liczba bijących ruchów gracza & 46 & . \\ 
 \hline
 17 & Rozmiar najdłuższego bijącego ruchu gracza & 47 & . \\
 \hline
 18 & Istnienie bijącego ruchu przeciwnika & 48 & . \\
 \hline
 19 & Liczba bijących ruchów przeciwnika & 49 & . \\
 \hline
 20 & Rozmiar najdłuższego bijącego ruchu przeciwnika & 50 & . \\
 \hline
 21 & Suma dystansów pionów gracza do rzędu awansu & 51 & . \\ 
 \hline
 22 & Suma dystansów pionów przeciwnika do rzędu awansu & 52 & . \\
 \hline
 23 & Liczba niezajętych pól w rzędzie awansu gracza & 53 & . \\
 \hline
 24 & Liczba niezajętych pól w rzędzie awansu przeciwnika & 54 & . \\
 \hline
 25 & . & 55 & . \\
 \hline
 26 & . & 56 & . \\ 
 \hline
 27 & . & 57 & . \\
 \hline
 28 & . & 58 & . \\
 \hline
 29 & . & 59 & . \\
 \hline
 30 & . & 60 & . \\
 \hline
\end{tabular}
}
\caption{Wszystkie parametry rozpatrywane w pracy. Część parametrów została zaczerpnięta z \cite{EvoCheckers}.}
\end{table}
\end{center}
}

\FloatBarrier

Wartość funkcji oceny heurystycznej można określić następującym wzorem:
{\centering
$\sum_{i=1}^{n}(param_i * weight_i)$
}

Funkcja oceny heurystycznej zależy oczywiście od podjętej strategii oraz od podejścia do problemu. Opisana wyżej funkcja ma parę zalet na których opiera się praca. Po pierwsze, samo wyznaczenie oceny z gotowymi wartościami parametrów odbywa się szybko, bo w czasie liniowym (nie uwzględnia to czasu potrzebnego do rozpatrzenia tychże parametrów). Po drugie, listowanie w ten sposób parametrów pozwoli na przeprowadzenie eksperymentów i wyciągnięcie wniosków o teorii gry w Warcaby, o czym w poniższym podrozdziale.

% \ldots WZÓR OBRAZUJĄCE DZIAŁANIE FUNKCJI \ldots

\section{Algorytm genetyczny}

Jednym z najważniejszych celów pracy jest znalezienie odpowiedniej strategii gry w Warcaby w postaci dobrej funkcji oceny heurystycznej, co w tym przypadku sprowadza się do wyznaczenia jak najlepszego zestawu wartości wag (dalej zwanego ciągiem wag). Naiwnym podejściem do tego problemu byłoby przypisanie priorytetów każdego parametru stanu planszy przez człowieka bądź zespół ludzi. Może się jednak okazać, że pewne parametry nie będą tak wartościowe dla zwycięstwa jak podpowiedziałaby ludzka intuicja. Ponadto, zakładając że nie istnieje obiektywne spojrzenie na wartość strategii grającej, należałoby przeprowadzić ogromną liczbę testów gry z człowiekiem w celu sprawdzenia poprawności wyznaczonych wag (tj. czy podane wartości ciągu wag są wystarczające by algorytm uznać za kompetytywnego gracza). Z pomocą tutaj przychodzi zastosowanie algorytmu genetycznego.

Algorytm genetyczny jest przedstawicielem klasy algorytmów metaheurystycznych. Konkretniej jest to odmiana algorytmu ewolucyjnego, który z kolei jest pochodną algorytmu populacyjnego. Metaheurystyka genetyczna symuluje zjawisko doboru naturalnego w przyrodzie, operując kolejnymi pokoleniami populacji osobników reprezentującymi potencjalne rozwiązania danego problemu, starając się odnaleźć rozwiązanie suboptymalne.

Najprostszy zarys algorytmu genetycznego można zobrazować w następujący sposób. W początkowej populacji losowo utworzonych osobników (rozwiązań) dochodzi do selekcji, mającej na celu wyznaczenie populacji rodziców. W populacji rodziców dochodzi do wzajemnego krzyżowania i tworzenia populacji dzieci, które dziedziczą po swoich potomkach informacje genetyczne (tj. elementy rozwiązania). Wśród dzieci może z pewnym prawdopodobieństwem dojść do mutacji, która losowo zmienia jeden z elementów genotypu u osobnika. W świeżo powstałej populacji powtarza się proces selekcji, krzyżowania i mutacji, aż do osiągnięcia z góry ustalonego warunku stopu (np. limit czasowy).

\ldots PRZYKŁAD PRZEBIEGU ALGORYTMU GENETYCZNEGO \ldots

Biorąc pod uwagę fakt, że to od rodziców zależy jakość każdej populacji, szczególny nacisk należy położyć na fazę selekcji. Niezbyt intuicyjną taktyką jest wprowadzanie w pewne miejsca przebiegu algorytmu elementów losowości, aby w populacji nie dochodziło do tak zwanego zjawiska stagnacji (sytuacja w której zbyt wiele osobników w populacji jest do siebie bardzo podobnych) oraz aby algorytm był w stanie znajdować inne, być może lepsze rozwiązania.

\subsection{Populacja i osobniki}

Osobnikiem populacji będzie ciąg wag funkcji oceny heurystycznej, reprezentowany jako tablica liczb całkowitych. Wagi będą mogły przyjmować zarówno dodatnie, jak i ujemne wartości.

\ldots RYSUNEK OBRAZUJĄCY OSOBNIKI W POPULACJI \ldots

\subsection{Selekcja i ewaluacja}

W wielu problemach do których stosuje się algorytmy genetyczne, funkcja ewaluacji osobnika jest podana w treści problemu bądź łatwa do wyznaczenia. Tak jednak nie jest z problemem znalezienia najlepiej grającej sztucznej inteligencji. Dlatego też zastosowane zostało podejście lokalne, czyli zwracające uwagę na umiejętności osobników w danej populacji. W procesie selekcji każdy ciąg wag gra z każdym innym ciągiem wag w podwójnym pojedynku (białe/czarne i czarne/białe). Im więcej gier wygra ciąg wag, tym wyższe prawdopodobieństwo że zostanie on wylosowany do populacji rodziców.

\subsection{Krzyżowanie i mutacja}

W pracy zaimplementowano dobieranie rodziców w pary w których się wzajemnie krzyżują, produkując dwójkę dzieci. Aby zachować potencjalnie dobre informacje genetyczne, dzieli się ciągi wag rodziców w tym samym losowo wyznaczonym miejscu i zamienia się ze sobą części. Dzięki temu każde z dwójki dzieci posiada cechy po każdym swoim rodzicu.

\ldots RYSUNEK KRZYŻOWANIA \ldots

Mutacja u nowo narodzonych osobników następuje z pewnym prawdopodobieństwem i wpływa na maksymalnie jedną wartość w ciągu wag. Mutowana wartość zostaje zastąpiona losową wartością z odpowiedniego przedziału.


