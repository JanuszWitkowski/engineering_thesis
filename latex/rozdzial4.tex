\chapter{Wyniki i rozszerzenia}
\thispagestyle{chapterBeginStyle}
\label{rozdzial4}

Można wyróżnić 2 główne cele eksperymentów w pracy: 
\begin{itemize}
\item \textbf{Sprawdzenie parametrów} - uruchomienie sesji algorytmu genetycznego w celu przypisania względnych wartości pod parametry w funkcji oceny heurystycznej;
\item \textbf{Porównanie perspektyw MIN i MAX} - przeprowadzenie dwóch sesji algorytmu genetycznego z~głębokościami różniącymi się od siebie o 1, a następnie porównanie wynikowych ciągów wag w celu ustalenia różnic między decyzjami gracza minimalizującego a decyzjami gracza maksymalizującego.
\end{itemize}

Do przeprowadzenia eksperymentów uruchomiono kilka sesji algorytmu genetycznego z różnymi argumentami. Poniżej znajdują się ich omówione wyniki.

\section{Sprawdzenie parametrów}

Przeprowadzono cztery różne sesje algorytmu genetycznego w celu znalezienia jak najlepszych priorytetów dla każdego parametru funkcji oceny heurystycznej. Ze względu na spory zakres wartości wag, losowość generowania wartości wag, jak i dużą liczbę parametrów, wprowadzono następujące oznaczenie priorytetów parametrów na podstawie wag:

\begin{itemize}
    \item \priohp $\rightarrow$ duża wartość wagi; parametr bardzo korzystny lub kluczowy dla gracza
    \item \priomp $\rightarrow$ umiarkowana wartość wagi; parametr korzystny dla gracza
    \item \priol $\rightarrow$ wartość wagi bliska zeru; algorytm nie zwraca uwagi na ten parametr
    \item \priomn $\rightarrow$ mała wartość wagi poniżej zera; parametr nieopłacalny dla gracza
    \item \priohn $\rightarrow$ bardzo niska wartość wagi; parametr staje się karą dla gracza
\end{itemize}

W tabeli~\ref{tab:results-params} przedstawione są parametry funkcji oceny heurystycznej wraz z uśrednionymi wartościami wag przetłumaczonych na podane wyżej priorytety.

\label{results-params}

{
% \small
\begin{center}
\begin{table}
\centering
{
\footnotesize
\begin{tabular}{|| c | c | c ||}
 \hline
 Nr & Parametr & Priorytet \\ %[0.5ex] 
 \hline\hline
 1 & Liczba sojuszniczych pionów & \priohn \\ 
 \hline
 2 & Liczba sojuszniczych damek & \priomp \\
 \hline
 3 & Liczba przeciwnych pionów & \priohp \\
 \hline
 4 & Liczba przeciwnych damek & \priohn \\
 \hline
 5 & Liczba sojuszniczych pionów przy ścianie & \priohp \\
 \hline
 6 & Liczba sojuszniczych damek przy ścianie & \priomn \\ 
 \hline
 7 & Liczba przeciwnych pionów przy ścianie & \priomn \\
 \hline
 8 & Liczba przeciwnych damek przy ścianie & \priohp \\
 \hline
 9 & Liczba ruchomych pionów gracza & \priomp \\
 \hline
 10 & Liczba ruchomych damek gracza & \priohp \\
 \hline
 11 & Liczba ruchomych pionów przeciwnika & \priohp \\ 
 \hline
 12 & Liczba ruchomych damek przeciwnika & \priomn \\
 \hline
 13 & Liczba możliwych ruchów gracza & \priohp \\
 \hline
 14 & Liczba możliwych ruchów przeciwnika & \priomp \\
 \hline
 15 & Istnienie bijącego ruchu gracza & \priomp \\
 \hline
 16 & Liczba bijących ruchów gracza & \priohn \\ 
 \hline
 17 & Rozmiar najdłuższego bijącego ruchu gracza & \priohn \\
 \hline
 18 & Istnienie bijącego ruchu przeciwnika & \priohn \\
 \hline
 19 & Liczba bijących ruchów przeciwnika & \priomp \\
 \hline
 20 & Rozmiar najdłuższego bijącego ruchu przeciwnika & \priohn \\
 \hline
 21 & Suma dystansów pionów gracza do rzędu awansu & \priomp \\ 
 \hline
 22 & Suma dystansów pionów przeciwnika do rzędu awansu & \priohn \\
 \hline
 23 & Liczba niezajętych pól w rzędzie awansu gracza & \priol \\
 \hline
 24 & Liczba niezajętych pól w rzędzie awansu przeciwnika & \priohn \\
 \hline
 25 & Liczba sojuszniczych pionów w dolnych rzędach & \priomn \\
 \hline
 26 & Liczba sojuszniczych damek w dolnych rzędach & \priomp \\ 
 \hline
 27 & Liczba przeciwnych pionów w dolnych rzędach & \priomp \\
 \hline
 28 & Liczba przeciwnych damek w dolnych rzędach & \priomp \\
 \hline
 29 & Liczba sojuszniczych pionów w środkowych rzędach & \priohn \\
 \hline
 30 & Liczba sojuszniczych damek w środkowych rzędach & \priohp \\
 \hline
 31 & Liczba przeciwnych pionów w środkowych rzędach & \priohn \\
 \hline
 32 & Liczba przeciwnych damek w środkowych wierszach & \priomn \\
 \hline
 33 & Liczba sojuszniczych pionów w górnych rzędach & \priomp \\
 \hline
 34 & Liczba sojuszniczych damek w górnych rzędach & \priohp \\
 \hline
 35 & Liczba przeciwnych pionów w górnych rzędach & \priomn \\
 \hline
 36 & Liczba przeciwnych damek w górnych rzędach & \priomp \\
 \hline
 37 & Liczba samotnych sojuszniczych pionów & \priol \\
 \hline
 38 & Liczba samotnych sojuszniczych damek & \priohn \\
 \hline
 39 & Liczba samotnych przeciwnych pionów & \priomn \\
 \hline
 40 & Liczba samotnych przeciwnych damek & \priohp \\
 \hline
 41 & Czy pion gracza jest w kącie & \priohn \\
 \hline
 42 & Czy damka gracza jest w kącie & \priohn \\
 \hline
 43 & Czy gracz zajmuje dwa kąty & \priomn \\
 \hline
 44 & Czy pion przeciwnika jest w kącie & \priol \\
 \hline
 45 & Czy damka przeciwnika jest w kącie & \priomn \\
 \hline
 46 & Czy przeciwnik zajmuje dwa kąty & \priohp \\
 \hline
 47 & Obecność \textit{Triangle pattern} u gracza & \priohn \\
 \hline
 48 & Obecność \textit{Oreo pattern} u gracza & \priohp \\
 \hline
 49 & Obecność \textit{Bridge pattern} u gracza & \priomn \\
 \hline
 50 & Obecność \textit{Dog pattern} u gracza & \priomp \\
 \hline
 51 & Obecność \textit{Triangle pattern} u przeciwnika & \priomp \\
 \hline
 52 & Obecność \textit{Oreo pattern} u przeciwnika & \priohn \\
 \hline
 53 & Obecność \textit{Bridge pattern} u przeciwnika & \priomn \\
 \hline
 54 & Obecność \textit{Dog pattern} u przeciwnika & \priohp \\
 \hline
 55 & Liczba blokujących sojuszniczych figur & \priol \\
 \hline
 56 & Liczba linii bloku gracza & \priomp \\
 \hline
 57 & Wielkość najdłuższej linii bloku gracza & \priomp \\
 \hline
 58 & Liczba blokujących przeciwnych figur & \priohp \\
 \hline
 59 & Liczba linii bloku przeciwnika & \priohn \\
 \hline
 60 & Wielkość najdłuższej linii bloku przeciwnika & \priomn \\
 \hline
\end{tabular}
}
\caption{Uśrednione wyniki kilku sesji algorytmu genetycznego.}
\label{tab:results-params}
\end{table}
\end{center}
}


Wynikowe priorytety części z parametrów mogą wydawać się nieintuicyjne (np. wartościowanie sojuszniczych pionów niżej niż przeciwnych pionów). Należy jednak pamiętać, że są one rezultatem przeprowadzenia ogromnej liczby turniejów z wykorzystaniem przeróżnych strategii. Ponadto, różne sesje algorytmu genetycznego potrafią znajdować różne rozwiązania, oraz, jak wspomniano w opisie algorytmu genetycznego~\ref{podrozdzial2-3}, nie można wykluczyć istnienia strategii rozmijających się z ludzkim pojmowaniem dobrych i złych decyzji w rozgrywce.

Należy zauważyć, że parametry liczby pionów powielają się w wielu innych parametrach, jak np. liczba pionów przy ściankach, liczba pionów w środkowych rzędach, liczba samotnych pionów. Mogło to sprawić, że licząc piony w wielu miejscach, algorytm musiał przypisać ujemną wagę ogólnej liczbie pionów, aby samo istnienie pionów nie przeważało w ostatecznej ocenie stanu. Oprócz tego, na późniejszych etapach gry piony mogą stać się dla gracza niekorzystne, choćby z tego powodu że lepiej jest mieć więcej damek. Warto sobie uświadomić, że obecna implementacja programu nie odróżnia etapu początkowego od etapu końcowego rozgrywki, przez co powstałą funkcję oceny heurystycznej można traktować jak uśrednienie taktyk opłacalnych na różnych stadiach gry.

Wynikowe priorytety parametrów 23, 37, 44 oraz 55 są niskiego rzędu, co sugeruje że nie mają one wpływu na jakość rozgrywki. Można je więc pominąć w ewaluacji stanów. Parametry o średnich priorytetach wywierają większą presję na ocenie heurystycznej i do pewnego stopnia warto zwracać na nie uwagę, jednak w ostatecznym rozrachunku może się okazać, że warto kierować się tylko parametrami o najwyższym priorytecie.

Na podstawie tych wyników i wniosków możemy zoptymalizować funkcję oceny heurystycznej, usuwając parametry których wpływ jest mniejszy w porównaniu do innych parametrów. Tabela~\ref{tab:results-params-opt} jest wynikiem wycięcia parametrów o niskim lub średnim priorytecie. Wprowadzono również wartości wag wyliczone w jednej z sesji eksperymentu.

\label{results-params-opt}

{
% \small
\begin{center}
\begin{table}
\centering
{
\footnotesize
% \small
\begin{tabular}{|| c | c | c ||}
 \hline
 Nr & Parametr & Waga \\ %[0.5ex] 
 \hline\hline
 1 & Liczba sojuszniczych pionów & -27848 \\ 
 \hline
 3 & Liczba przeciwnych pionów & 22520 \\
 \hline
 % not from 3ga3
 4 & Liczba przeciwnych damek & -24912 \\
 \hline
 5 & Liczba sojuszniczych pionów przy ścianie & 25644 \\
 \hline
 8 & Liczba przeciwnych damek przy ścianie & 27337 \\
 \hline
 10 & Liczba ruchomych damek gracza & 4708 \\
 \hline
 11 & Liczba ruchomych pionów przeciwnika & 6119 \\ 
 \hline
 13 & Liczba możliwych ruchów gracza & 21613 \\
 \hline
 16 & Liczba bijących ruchów gracza & -28692 \\ 
 \hline
 17 & Rozmiar najdłuższego bijącego ruchu gracza & -20989 \\
 \hline
 18 & Istnienie bijącego ruchu przeciwnika & -20551 \\
 \hline
 20 & Rozmiar najdłuższego bijącego ruchu przeciwnika & -13876 \\
 \hline
 22 & Suma dystansów pionów przeciwnika do rzędu awansu & -24871 \\
 \hline
 24 & Liczba niezajętych pól w rzędzie awansu przeciwnika & -30665 \\
 \hline
 29 & Liczba sojuszniczych pionów w środkowych rzędach & -24696 \\
 \hline
 30 & Liczba sojuszniczych damek w środkowych rzędach & 26462 \\
 \hline
 31 & Liczba przeciwnych pionów w środkowych rzędach & -32375 \\
 \hline
 34 & Liczba sojuszniczych damek w górnych rzędach & 27159 \\
 \hline
 38 & Liczba samotnych sojuszniczych damek & -20871 \\
 \hline
 40 & Liczba samotnych przeciwnych damek & 11132 \\
 \hline
 41 & Czy pion gracza jest w kącie & -25605 \\
 \hline
 % not from 3ga3
 42 & Czy damka gracza jest w kącie & -32651 \\
 \hline
 46 & Czy przeciwnik zajmuje dwa kąty & 25836 \\
 \hline
 47 & Obecność \textit{Triangle pattern} u gracza & -25839 \\
 \hline
 48 & Obecność \textit{Oreo pattern} u gracza & 1326 \\
 \hline
 52 & Obecność \textit{Oreo pattern} u przeciwnika & -618 \\
 \hline
 54 & Obecność \textit{Dog pattern} u przeciwnika & 14026 \\
 \hline
 58 & Liczba blokujących przeciwnych figur & 32446 \\
 \hline
 59 & Liczba linii bloku przeciwnika & -28419 \\
 \hline
\end{tabular}
}
\caption{Zestaw priorytetowych parametrów z wartościami wag}
\label{tab:results-params-opt}
\end{table}
\end{center}
}


 % 1 & Liczba sojuszniczych pionów & \priohn \\ 
 % \hline
 % 3 & Liczba przeciwnych pionów & \priohp \\
 % \hline
 % 4 & Liczba przeciwnych damek & \priohn \\
 % \hline
 % 5 & Liczba sojuszniczych pionów przy ścianie & \priohp \\
 % \hline
 % 8 & Liczba przeciwnych damek przy ścianie & \priohp \\
 % \hline
 % 10 & Liczba ruchomych damek gracza & \priohp \\
 % \hline
 % 11 & Liczba ruchomych pionów przeciwnika & \priohp \\ 
 % \hline
 % 13 & Liczba możliwych ruchów gracza & \priohp \\
 % \hline
 % 16 & Liczba bijących ruchów gracza & \priohn \\ 
 % \hline
 % 17 & Rozmiar najdłuższego bijącego ruchu gracza & \priohn \\
 % \hline
 % 18 & Istnienie bijącego ruchu przeciwnika & \priohn \\
 % \hline
 % 20 & Rozmiar najdłuższego bijącego ruchu przeciwnika & \priohn \\
 % \hline
 % 22 & Suma dystansów pionów przeciwnika do rzędu awansu & \priohn \\
 % \hline
 % 24 & Liczba niezajętych pól w rzędzie awansu przeciwnika & \priohn \\
 % \hline
 % 29 & Liczba sojuszniczych pionów w środkowych rzędach & \priohn \\
 % \hline
 % 30 & Liczba sojuszniczych damek w środkowych rzędach & \priohp \\
 % \hline
 % 31 & Liczba przeciwnych pionów w środkowych rzędach & \priohn \\
 % \hline
 % 34 & Liczba sojuszniczych damek w górnych rzędach & \priohp \\
 % \hline
 % 38 & Liczba samotnych sojuszniczych damek & \priohn \\
 % \hline
 % 40 & Liczba samotnych przeciwnych damek & \priohp \\
 % \hline
 % 41 & Czy pion gracza jest w kącie & \priohn \\
 % \hline
 % 42 & Czy damka gracza jest w kącie & \priohn \\
 % \hline
 % 46 & Czy przeciwnik zajmuje dwa kąty & \priohp \\
 % \hline
 % 47 & Obecność \textit{Triangle pattern} u gracza & \priohn \\
 % \hline
 % 48 & Obecność \textit{Oreo pattern} u gracza & \priohp \\
 % \hline
 % 52 & Obecność \textit{Oreo pattern} u przeciwnika & \priohn \\
 % \hline
 % 54 & Obecność \textit{Dog pattern} u przeciwnika & \priohp \\
 % \hline
 % 58 & Liczba blokujących przeciwnych figur & \priohp \\
 % \hline
 % 59 & Liczba linii bloku przeciwnika & \priohn \\
 % \hline


Jak już wcześniej wspomniano, niska waga liczby sojuszniczych pionów może być wynikiem przewrażliwienia funkcji oceny ze względu na wysoką liczebność parametrów rozpatrujących piony, chociażby parametr 5 (obejmuje on piony przy ścianach, czyli takie których nie da się zbić). Wysoka waga liczby przeciwnych pionów (parametry 3, 11) może dodatkowo wskazywać, że na pewnym etapie gry w warcaby posiadanie pionów jest postrzegane jako słaby punkt na planszy. W tym przekonaniu utwierdza ujemna wartość wagi dla przeciwnych damek - dla gracza lepiej aby przeciwnik transformował jak najmniej pionów w damki.

Z obserwacji rezultatów wynika, że algorytm lubi mieć swobodę w wykonywaniu ruchów, lecz nie przepada za sytuacjami w których ma wymuszone bicie, najprawdopodobniej dlatego że jest poza jego kontrolą. Nie jest to jednak własność symetryczna, ponieważ nie przepada też za tym, gdy jego oponent ma w danej chwili ruch bijący. Potencjalną hipotezą tłumaczącą to zjawisko jest różnica między traceniem figur a ich transformacją - algorytm woli zmniejszyć liczbę swoich pionów awansując je do damek, zamiast dać im się zbić przeciwnikowi.

Priorytet parametru 22 może wskazywać na to, że algorytm woli gdy piony przeciwnika znajdują się bliżej ,,bazy'' gracza, być może dlatego, że gracz może je wówczas łatwiej zbić. Możliwym jest też, że parametr sumy dystansów pionów przeciwnika do rzędu awansu w pewien sposób preferuje stany w których przeciwnik posiada ogólnie mało pionów (wówczas wartość tego parametru też jest mała). Argumentem za tą hipotezą może być waga parametru 24, która silnie przeciwstawia się sposobności oponenta do awansu jego pionów.

Dalej można wnioskować że, według wyznaczonych wag, graczowi nie opłaca się zostawiać pionów na środku planszy, ale za to bardzo mu się opłaca przeprowadzać je wzdłuż ścian planszy i awansować je by umiejscowić damki w ,,bazie'' przeciwnika. Równocześnie damki okazują się być figurami niewykorzystywanymi w pełni, jeżeli nie mają innych figur w pobliżu. Może to świadczyć o tym, że damka przynosi najlepsze rezultaty, gdy wywiera ciągłą presję na przeciwniku, lub że damki należy chronić innymi sojuszniczymi figurami, by nie zostały zbite.

Jeżeli chodzi o parametry binarne, można wnioskować że algorytm uważa figury siedzące w kątach za bezużyteczne i przynoszące niekorzyść ich właścicielowi, podobnie dla pionów wchodzących w skład \textit{Triangle pattern}. Pozytywna waga \textit{Oreo pattern} może być powiązana ze wcześniej przeanalizowanym parametrem 24. Warto zauważyć, że jeśli istnieje u przeciwnika \textit{Dog pattern}, gracz posiada niemożliwego do zbicia piona, który albo blokuje jednego piona przeciwnika, albo ma prostą drogę do awansu. Możliwym jest, że to właśnie stąd wygenerowana została pozytywna waga takiej sytuacji.

Ostatnie dwa parametry w tabeli~\ref{tab:results-params-opt} można wytłumaczyć następującą hipotezą. Algorytm uważa, że blokujące figury nie są użyteczne ze względu na brak mobilności, być może dlatego że zwraca większą uwagę na ofensywną strategię awansowania pionów. Defensywne linie bloku oponenta przeszkadzają mu w tym jednak, dlatego też woli unikać takich stanów.

Powstałą strategię algorytmu można podsumować jako agresywną taktykę dążącą do jak najszybszego utworzenia damek, przy jednoczesnym uniemożliwieniu awansu przeciwnikowi. Algorytm widzi olbrzymi potencjał w damce, która notabene jest najpotężniejszą figurą w grze. Dzięki niej gracz może powybijać piony przeciwnika od wewnątrz.

\section{Porównanie perspektyw MIN i MAX}
\label{sub:min_max}

W ramach eksperymentu przeprowadzono cztery różne sesje algorytmu genetycznego, z czego w dwóch sesjach głębokość przeszukiwań wynosiła 4, a w dwóch pozostałych głębokość wynosiła 5. Celem eksperymentu było sprawdzenie, czy funkcja oceny heurystycznej zależy od tego, kto podejmuje decyzje przy liściach drzewa przeszukiwań w Minimaksie - gracz MAX czy gracz MIN. W~tabelach~\ref{tab:results-minmax1} oraz~\ref{tab:results-minmax2} znajdują się wyniki eksperymentu.

\label{results-minmax1}

{
% \small
\begin{center}
\begin{table}
\centering
{
\footnotesize
\begin{tabular}{|| c | c | c | c ||}
 \hline
 Nr & Parametr & Priorytet h = 4 & Priorytet h = 5 \\ %[0.5ex] 
 \hline\hline
 1 & Liczba sojuszniczych pionów & \priohp & \priohp \\ 
 \hline
 2 & Liczba sojuszniczych damek & \priohp & \priohp \\
 \hline
 3 & Liczba przeciwnych pionów & \priomn & \priohn \\
 \hline
 4 & Liczba przeciwnych damek & \priohn & \priohn \\
 \hline
 5 & Liczba sojuszniczych pionów przy ścianie & \priol & \priohp \\
 \hline
 6 & Liczba sojuszniczych damek przy ścianie & \priomp & \priohp \\ 
 \hline
 7 & Liczba przeciwnych pionów przy ścianie & \priomn & \priol \\
 \hline
 8 & Liczba przeciwnych damek przy ścianie & \priohn & \priohn \\
 \hline
 9 & Liczba ruchomych pionów gracza & \priol & \priomp \\
 \hline
 10 & Liczba ruchomych damek gracza & \priohp & \priohp \\
 \hline
 11 & Liczba ruchomych pionów przeciwnika & \priomn & \priomn \\ 
 \hline
 12 & Liczba ruchomych damek przeciwnika & \priohn & \priohn \\
 \hline
 13 & Liczba możliwych ruchów gracza & \priomp & \priomn \\
 \hline
 14 & Liczba możliwych ruchów przeciwnika & \priomp & \priomn \\
 \hline
 15 & Istnienie bijącego ruchu gracza & \priohp & \priomp \\
 \hline
 16 & Liczba bijących ruchów gracza & \priohn & \priohp \\ 
 \hline
 17 & Rozmiar najdłuższego bijącego ruchu gracza & \priohp & \priomp \\
 \hline
 18 & Istnienie bijącego ruchu przeciwnika & \priomn & \priohn \\
 \hline
 19 & Liczba bijących ruchów przeciwnika & \priomn & \priol \\
 \hline
 20 & Rozmiar najdłuższego bijącego ruchu przeciwnika & \priomp & \priohn \\
 \hline
 21 & Suma dystansów pionów gracza do rzędu awansu & \priohp & \priohp \\ 
 \hline
 22 & Suma dystansów pionów przeciwnika do rzędu awansu & \priohn & \priohn \\
 \hline
 23 & Liczba niezajętych pól w rzędzie awansu gracza & \priomp & \priomn \\
 \hline
 24 & Liczba niezajętych pól w rzędzie awansu przeciwnika & \priohn & \priohn \\
 \hline
 25 & Liczba sojuszniczych pionów w dolnych rzędach & \priomn & \priohn \\
 \hline
 26 & Liczba sojuszniczych damek w dolnych rzędach & \priohn & \priohp \\ 
 \hline
 27 & Liczba przeciwnych pionów w dolnych rzędach & \priohp & \priol \\
 \hline
 28 & Liczba przeciwnych damek w dolnych rzędach & \priol & \priomp \\
 \hline
 29 & Liczba sojuszniczych pionów w środkowych rzędach & \priomp & \priomp \\
 \hline
 30 & Liczba sojuszniczych damek w środkowych rzędach & \priohp & \priohp \\
 \hline
 31 & Liczba przeciwnych pionów w środkowych rzędach & \priohn & \priohn \\
 \hline
 32 & Liczba przeciwnych damek w środkowych wierszach & \priomp & \priomp \\
 \hline
 33 & Liczba sojuszniczych pionów w górnych rzędach & \priohp & \priohp \\
 \hline
 34 & Liczba sojuszniczych damek w górnych rzędach & \priohp & \priohp \\
 \hline
 35 & Liczba przeciwnych pionów w górnych rzędach & \priohn & \priohn \\
 \hline
 36 & Liczba przeciwnych damek w górnych rzędach & \priohn & \priol \\
 \hline
 37 & Liczba samotnych sojuszniczych pionów & \priohp & \priohp \\
 \hline
 38 & Liczba samotnych sojuszniczych damek & \priomp & \priomp \\
 \hline
 39 & Liczba samotnych przeciwnych pionów & \priohn & \priomn \\
 \hline
 40 & Liczba samotnych przeciwnych damek & \priohn & \priohn \\
 \hline
 41 & Czy pion gracza jest w kącie & \priohn & \priohn \\
 \hline
 42 & Czy damka gracza jest w kącie & \priomn & \priohn \\
 \hline
 43 & Czy gracz zajmuje dwa kąty & \priomp & \priohn \\
 \hline
 44 & Czy pion przeciwnika jest w kącie & \priohp & \priohp \\
 \hline
 45 & Czy damka przeciwnika jest w kącie & \priomp & \priomp \\
 \hline
 46 & Czy przeciwnik zajmuje dwa kąty & \priomp & \priomp \\
 \hline
 47 & Obecność \textit{Triangle pattern} u gracza & \priohp & \priohp \\
 \hline
 48 & Obecność \textit{Oreo pattern} u gracza & \priohn & \priomp \\
 \hline
 49 & Obecność \textit{Bridge pattern} u gracza & \priomp & \priohp \\
 \hline
 50 & Obecność \textit{Dog pattern} u gracza & \priol & \priohp \\
 \hline
 51 & Obecność \textit{Triangle pattern} u przeciwnika & \priol & \priohn \\
 \hline
 52 & Obecność \textit{Oreo pattern} u przeciwnika & \priomn & \priol \\
 \hline
 53 & Obecność \textit{Bridge pattern} u przeciwnika & \priol & \priol \\
 \hline
 54 & Obecność \textit{Dog pattern} u przeciwnika & \priohp & \priohp \\
 \hline
 55 & Liczba blokujących sojuszniczych figur & \priohp & \priohp \\
 \hline
 56 & Liczba linii bloku gracza & \priomp & \priohp \\
 \hline
 57 & Wielkość najdłuższej linii bloku gracza & \priol & \priomp \\
 \hline
 58 & Liczba blokujących przeciwnych figur & \priohn & \priomn \\
 \hline
 59 & Liczba linii bloku przeciwnika & \priol & \priol \\
 \hline
 60 & Wielkość najdłuższej linii bloku przeciwnika & \priomp & \priomp \\
 \hline
\end{tabular}
}
\caption{Porównanie priorytetów dla głębokości 4 oraz głębokości 5, sesja 1.}
\label{tab:results-minmax1}
\end{table}
\end{center}
}


\label{results-minmax2}

{
% \small
\begin{center}
\begin{table}
\centering
{
\footnotesize
\begin{tabular}{|| c | c | c | c ||}
 \hline
 Nr & Parametr & Priorytet h = 4 & Priorytet h = 5 \\ %[0.5ex] 
 \hline\hline
 1 & Liczba sojuszniczych pionów & \priohp & \priomn \\ 
 \hline
 2 & Liczba sojuszniczych damek & \priohn & \priomn \\
 \hline
 3 & Liczba przeciwnych pionów & \priomn & \priohn \\
 \hline
 4 & Liczba przeciwnych damek & \priol & \priohp \\
 \hline
 5 & Liczba sojuszniczych pionów przy ścianie & \priohp & \priomp \\
 \hline
 6 & Liczba sojuszniczych damek przy ścianie & \priohp & \priomn \\ 
 \hline
 7 & Liczba przeciwnych pionów przy ścianie & \priohn & \priomp \\
 \hline
 8 & Liczba przeciwnych damek przy ścianie & \priohn & \priomp \\
 \hline
 9 & Liczba ruchomych pionów gracza & \priohp & \priohp \\
 \hline
 10 & Liczba ruchomych damek gracza & \priomn & \priomn \\
 \hline
 11 & Liczba ruchomych pionów przeciwnika & \priomn & \priohn \\ 
 \hline
 12 & Liczba ruchomych damek przeciwnika & \priomp & \priohn \\
 \hline
 13 & Liczba możliwych ruchów gracza & \priomp & \priomp \\
 \hline
 14 & Liczba możliwych ruchów przeciwnika & \priohp & \priohp \\
 \hline
 15 & Istnienie bijącego ruchu gracza & \priohn & \priomn \\
 \hline
 16 & Liczba bijących ruchów gracza & \priohn & \priohn \\ 
 \hline
 17 & Rozmiar najdłuższego bijącego ruchu gracza & \priohn & \priomp \\
 \hline
 18 & Istnienie bijącego ruchu przeciwnika & \priomp & \priohn \\
 \hline
 19 & Liczba bijących ruchów przeciwnika & \priohn & \priohp \\
 \hline
 20 & Rozmiar najdłuższego bijącego ruchu przeciwnika & \priohn & \priohn \\
 \hline
 21 & Suma dystansów pionów gracza do rzędu awansu & \priohp & \priomp \\ 
 \hline
 22 & Suma dystansów pionów przeciwnika do rzędu awansu & \priohn & \priohn \\
 \hline
 23 & Liczba niezajętych pól w rzędzie awansu gracza & \priol & \priomn \\
 \hline
 24 & Liczba niezajętych pól w rzędzie awansu przeciwnika & \priohp & \priohp \\
 \hline
 25 & Liczba sojuszniczych pionów w dolnych rzędach & \priohp & \priol \\
 \hline
 26 & Liczba sojuszniczych damek w dolnych rzędach & \priomp & \priomp \\ 
 \hline
 27 & Liczba przeciwnych pionów w dolnych rzędach & \priol & \priol \\
 \hline
 28 & Liczba przeciwnych damek w dolnych rzędach & \priomn & \priomn \\
 \hline
 29 & Liczba sojuszniczych pionów w środkowych rzędach & \priohp & \priohn \\
 \hline
 30 & Liczba sojuszniczych damek w środkowych rzędach & \priomp & \priomp \\
 \hline
 31 & Liczba przeciwnych pionów w środkowych rzędach & \priol & \priomn \\
 \hline
 32 & Liczba przeciwnych damek w środkowych wierszach & \priomn & \priomp \\
 \hline
 33 & Liczba sojuszniczych pionów w górnych rzędach & \priomp & \priomp \\
 \hline
 34 & Liczba sojuszniczych damek w górnych rzędach & \priohn & \priohn \\
 \hline
 35 & Liczba przeciwnych pionów w górnych rzędach & \priomp & \priohp \\
 \hline
 36 & Liczba przeciwnych damek w górnych rzędach & \priohp & \priol \\
 \hline
 37 & Liczba samotnych sojuszniczych pionów & \priol & \priohn \\
 \hline
 38 & Liczba samotnych sojuszniczych damek & \priohp & \priohn \\
 \hline
 39 & Liczba samotnych przeciwnych pionów & \priomp & \priomn \\
 \hline
 40 & Liczba samotnych przeciwnych damek & \priohp & \priomp \\
 \hline
 41 & Czy pion gracza jest w kącie & \priohp & \priomp \\
 \hline
 42 & Czy damka gracza jest w kącie & \priomn & \priohn \\
 \hline
 43 & Czy gracz zajmuje dwa kąty & \priomn & \priol \\
 \hline
 44 & Czy pion przeciwnika jest w kącie & \priomn & \priomp \\
 \hline
 45 & Czy damka przeciwnika jest w kącie & \priohn & \priomn \\
 \hline
 46 & Czy przeciwnik zajmuje dwa kąty & \priohp & \priomn \\
 \hline
 47 & Obecność \textit{Triangle pattern} u gracza & \priohp & \priomp \\
 \hline
 48 & Obecność \textit{Oreo pattern} u gracza & \priomn & \priohp \\
 \hline
 49 & Obecność \textit{Bridge pattern} u gracza & \priol & \priomn \\
 \hline
 50 & Obecność \textit{Dog pattern} u gracza & \priomp & \priomn \\
 \hline
 51 & Obecność \textit{Triangle pattern} u przeciwnika & \priol & \priohn \\
 \hline
 52 & Obecność \textit{Oreo pattern} u przeciwnika & \priomp & \priomp \\
 \hline
 53 & Obecność \textit{Bridge pattern} u przeciwnika & \priomp & \priohn \\
 \hline
 54 & Obecność \textit{Dog pattern} u przeciwnika & \priohp & \priohn \\
 \hline
 55 & Liczba blokujących sojuszniczych figur & \priohp & \priol \\
 \hline
 56 & Liczba linii bloku gracza & \priol & \priol \\
 \hline
 57 & Wielkość najdłuższej linii bloku gracza & \priol & \priohp \\
 \hline
 58 & Liczba blokujących przeciwnych figur & \priohp & \priohp \\
 \hline
 59 & Liczba linii bloku przeciwnika & \priol & \priomp \\
 \hline
 60 & Wielkość najdłuższej linii bloku przeciwnika & \priol & \priomn \\
 \hline
\end{tabular}
}
\caption{Porównanie priorytetów dla głębokości 4 oraz głębokości 5, sesja 2.}
\label{tab:results-minmax2}
\end{table}
\end{center}
}



Można zaobserwować, że mimo wielu różnic, w obu przypadkach obie głębokości poskutkowały tymi samymi priorytetami dla około jednej trzeciej parametrów. Obie perspektywy (MIN i MAX) przykładają mniejszą uwagę do parametrów związanych z \textit{patternami}. Można przypuszczać, że w takim razie istnieją uniwersalne i niezależne od perspektywy parametry, których priorytety są niezmienne.

Liczność par parametrów o różniących się priorytetach skłania jednak do wniosku, że gracz MIN inaczej rozpatruje stany gry od gracza MAX. Jeśli założymy że algorytm genetyczny dąży zawsze do tego samego rozwiązania z tych samych początkowych danych (co wcale nie musi być prawdą), to rezultaty eksperymentu mogą być empirycznym dowodem na to, że sam fakt wyboru innego ekstremalnego stanu z poddrzewa stanów na maksymalnej głębokości przeszukiwań prowadzi do konieczności obrania trochę innej strategii. Z~ogólnej obserwacji wyników można zaryzykować stwierdzenie, że taktyka wygenerowana w sesji w której MAX decydował jako pierwszy jest bardziej ofensywna, natomiast w sesji w której to MIN decydował pierwszy - bardziej defensywna (wyższe wartościowanie liczby pionów oraz figur blokujących).

\section{Możliwości rozwoju projektu}

Mimo osiągnięcia zamierzonych celów wciąż pozostaje kilka aspektów pracy, które da się rozwinąć lub ulepszyć. Otwarcie kodów źródłowych na rozszerzenia może ułatwić dodanie nowych funkcjonalności bądź modyfikację kopii niektórych klas. Poniżej znajduje się kilka propozycji rozwinięcia projektu dalej.

\subsection{Optymalizacje}

Przeszukiwanie przestrzeni stanów w Minimaksie można usprawnić o bazę rozpoczęć i zakończeń - algorytm mógłby zaczynać gry lub odpowiadać jednym ze standardowych rozpoczęć turniejowych i, gdy nadarzy się sposobność, dążyć do jednego z zakończeń. Na podobnej zasadzie działał np. Deep Blue, komputer firmy IBM, który jako pierwsza maszyna na świecie zwyciężył w partii szachów z ówczesnym mistrzem świata Garri Kasparowem w 1996 roku \cite{RBA-SI}.

Innym pomysłem na potencjalne skrócenie obliczeń jest zastosowanie Hashmapy przeszukanych stanów z przypisanymi im ocenami. Pomysł ten bazuje na obserwacji, że do niektórych stanów na planszy można dojść z kilku innych stanów. Rozpatrując stan, algorytm sięgałby do takiej Hashmapy i jeżeli znalazłby hash tego stanu, automatycznie przyznawałby ocenę bez konieczności schodzenia głębiej w drzewie. Warto jednak zaznaczyć, że operacje dodania i przeszukania w Hashmapie nie są stałe (zajmują złożoność logarytmiczną zależną od liczby stanów) i wykonywane są dla każdego stanu, a samych stanów może być w~pewnym momencie bardzo dużo.

Jeszcze innym aspektem, którego optymalizacja mogłaby znacznie poprawić wydajność, szczególnie dla sesji algorytmu genetycznego, są funkcje analizujące planszę i wyliczające parametry do oceny heurystycznej. W momencie pisania pracy istnieje wiele funkcji wyliczających wartość jednego parametru w czterech wariantach (sojusznicze piony, sojusznicze damki, przeciwne piony, przeciwne damki). Warto jednak zwrócić uwagę na fakt, że prawie każda z tych funkcji musi przeanalizować całą planszę po każdym polu. Stąd pomysł na optymalizację: utworzyć specjalną klasę \textit{StateAnalyzer} przechowującą informacje o~wszystkich parametrach oraz booleanową flagę wskazującą, czy informacje te są aktualne. Przy każdym ruchu flagę ustawia się na \textit{false}, a w razie konieczności (gdy zewnętrzny obiekt prosi o wartość parametru) ustawia się flagę na \textit{true} i oblicza wartości każdego parametru od razu. Innym sposobem na optymalizację w tym zakresie jest obliczanie na nowo wartości tylko tych parametrów, które rzeczywiście zmieniają się w danym ruchu.

Dodatkowo można spróbować ulepszyć wydajność samego algorytmu genetycznego. Łatwo zauważyć, że do wybrania lepiej przystosowanej połowy populacji osobników (z czynnikiem losowym) nie potrzeba sortowania gorzej przystosowanej połowy populacji, ponieważ ta zostaje odrzucona. Sortowanie tylko części populacji miałoby szansę wywrzeć zauważalny wpływ na czas wykonania obliczeń. Można też eksperymentować z wprowadzaniem innych algorytmów sortujących, chociażby \textit{QuickSort}.

\subsection{Walka z efektem horyzontu}

Efektem horyzontu nazywamy problem, w którym ograniczenie na wielkość przeszukiwanego kawałka przestrzeni stanów uniemożliwia dojście do potencjalnie lepszego rozwiązania znajdującego się krok dalej. Obecna implementacja Minimaxa w pracy jest podatna na problem horyzontu. Można częściowo temu zapobiec, zmuszając algorytm do rozpatrzenia dzieci stanu na maksymalnej głębokości, jeżeli jest on jedynym stanem pochodnym swojego rodzica (możliwość wykonania tylko jednego ruchu najczęściej oznacza wymuszone bicie, które jest poza kontrolą gracza). Można oszacować koszt takiego sprawdzenia jako większy zaledwie o jeden od średniego kosztu przejścia drzewa przeszukiwań - to tak jakby przenieść jedno poddrzewo o rząd niżej i umieścić w jego miejsce jeden stan. Ideę tę rozwija ,,przeszukiwanie uspokajające'' (\textit{Quiescence Search} \cite{QSearch}), które zatrzymuje przeszukiwanie poddrzew tylko na stanach spokojnych, czyli na takich które w najbliższych paru krokach nie zmienią się drastycznie (np. w warcabach stany tuż przed biciem lub awansem piona do damki nie są spokojne).

\subsection{Analiza MINa i MAXa}

Jak wspomniano we wnioskach podrozdziału~\ref{sub:min_max}, rezultaty badań dwóch perspektyw w Minimaksie sugerują, że różnica między ocenami których bezpośrednie wartości dla stanów są minimalizowane, a ocenami których wartości są maksymalizowane, jest niemała. Być może istnieją pewne niuanse związane ze spojrzeniem graczy na maksymalnej głębokości przeszukiwania. W ramach rozwinięcia pracy można by było przyjrzeć się temu zjawisku, przebadać je i zapisać nowe wnioski.

\subsection{Interfejs}

W obecnej chwili program prowadzący rozgrywki w warcaby uruchamiany jest z poziomu konsoli. Miłym rozszerzeniem byłoby stworzenie przyjaznego ludzkiemu użytkownikowi interfejsu do uruchamiania rozgrywek oraz intuicyjnego wprowadzania ruchów, chociażby poprzez kliknięcia. Nie zaszkodzą również dodatkowe opcje, np. możliwość cofnięcia ruchu, zapis i odczyt rozgrywki, przechowywanie interaktywnej historii rozgrywek. To wszystko można opakować w stronę webową i ją udostępnić.

Ciekawym pomysłem jest również napisanie adaptera, który pozwalałby sztucznej inteligencji stworzonej w pracy prowadzić rozgrywki z innymi dostępnymi graczami komputerowymi w internecie. Pozwoliłoby to na poznanie siły wyznaczonej sztucznej inteligencji.

