\begin{footnotesize}
\begin{lstlisting}[language=Java, frame=lines, numberstyle=\tiny, stepnumber=5, caption=Metoda selekcji w algorytmie genetycznym, firstnumber=1]
private short[][] selection (short[][] population) {
    int popSize = population.length;
    // 0 - index; 1 - wygrane w ataku; 2 - wygrane w obronie; 3 - ogólny wynik.
    int[][] results = playDuels(population, popSize);
    // Znajdź najlepszy wynik
    for (int j = popSize - 1; j > 0; --j) {
        if (results[j-1][3] < results[j][3]) {
            int[] tmp = results[j];
            results[j] = results[j-1];
            results[j-1] = tmp;
        }
    }
    // Wybierz najlepszego.
    if (bestSoFar == null) bestSoFar = population[results[0][0]];
    else {  // Przeprowadź pojedynek gladiatorski
        short[] bestThisTime = population[results[0][0]];
        player1.changeHeuristicWeights(bestThisTime);
        player2.changeHeuristicWeights(bestSoFar);
        game1.resetBoard();
        game2.resetBoard();
        int result1 = game1.quickGame();
        int result2 = (-1) * game2.quickGame();
        if (result1 + result2 >= 0) bestSoFar = bestThisTime;
    }
    // Ruletka, czyli loteria osobników które przechodzą dalej.
    for (int i = 0; i < popSize; ++i) {
        results[i][3] += RNG.randomInt(selectionFactor);
    }
    // Insertion sort
    for (int i = 1; i < popSize; ++i) {
        for (int j = i; j > 0; --j) {
            if (results[j-1][3] < results[j][3]) {
                int[] tmp = results[j];
                results[j] = results[j-1];
                results[j-1] = tmp;
            } else break;
        }
    }
    // W miarę możliwości dobieraj osobniki różne.
    boolean[] candidatesFree = new boolean[popSize];
    for (int i = 0; i < popSize; ++i) candidatesFree[i] = true;
    short[][] parents = new short[parentPopulationSize][genotypeSize];
    int numberOfParents = 0;
    for (int i = 0; i < popSize && numberOfParents < parentPopulationSize; ++i) {
        short[] candidate = population[results[i][0]];
        if (isGenotypeNotInPopulation(candidate, parents, 0, numberOfParents)) {
            parents[numberOfParents] = candidate;
            ++numberOfParents;
            candidatesFree[i] = false;
        }
    }
    // Dokończ populację rodziców duplikatami, aby nie było pustych miejsc.
    int index = 0;
    while (numberOfParents < parentPopulationSize) {
        short[] candidate = population[results[index][0]];
        if (candidatesFree[index]) {
            parents[numberOfParents] = candidate;
            ++numberOfParents;
            candidatesFree[index] = false;
        }
        ++index;
    }
    return parents;
}
\end{lstlisting} 
\end{footnotesize}

