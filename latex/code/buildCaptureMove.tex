\begin{footnotesize}
\begin{lstlisting}[language=Java, frame=lines, numberstyle=\tiny, stepnumber=5, caption=Metoda w klasie State budująca drzewo możliwych bić, firstnumber=1]
private void buildCaptureMove (Node parent, int height, ArrayList<ArrayList<Integer>> moves,
        int row, int col, int dr, boolean isKing) {
    Node next = new Node(coordinatesToNumber(row, col), height + 1, parent);
    int adjRow, adjCol, newRow, newCol;
    boolean nodeIsLeaf = true, checkOtherRow = !isKing;
    do {
        checkOtherRow = !checkOtherRow;
        for (int dc = -1; dc <= 1; dc += 2) {
            adjRow = row + dr;
            adjCol = col + dc;
            if (isInsideTheBoard(adjRow, adjCol)) {
                int owner = ownerOfField(adjRow, adjCol);
                if (owner != 0 && owner != ownerOfField(row, col)) {
                    newRow = adjRow + dr;
                    newCol = adjCol + dc;
                    if (isInsideTheBoard(newRow, newCol) && board[newRow][newCol] == 0) {
                        nodeIsLeaf = false;
                        board[newRow][newCol] = board[row][col];
                        board[row][col] = 0;
                        int capturedPiece = board[adjRow][adjCol];
                        board[adjRow][adjCol] = 0;
                        buildCaptureMove(next, next.height(), moves,
                            newRow, newCol, dr, isKing);
                        board[adjRow][adjCol] = capturedPiece;
                        board[row][col] = board[newRow][newCol];
                        board[newRow][newCol] = 0;
                    }
                }
            }
        }
        dr *= -1;
    } while (checkOtherRow);
    if (nodeIsLeaf) {
        ArrayList<Integer> move = getMoveFromTree(next);
        if (!move.isEmpty()) {
            moves.add(move);
        }
    }
}
\end{lstlisting} 
\end{footnotesize}
