%Korekta ALD - nienumerowany wstęp
%\chapter{Wstęp}
\addcontentsline{toc}{chapter}{Wstęp}
\chapter*{Wstęp}

\thispagestyle{chapterBeginStyle}

Tematem pracy jest prosty model sztucznej inteligencji do gry w warcaby w wariancie angielskim, oparty o algorytm decyzyjny Minimax oraz funkcję oceny heurystycznej stanu gry. Celem pracy było wyznaczenie jak najlepszej funkcji oceny stanowiącej intuicję algorytmu, oraz sprawdzenie różnic między możliwymi do obrania w~Minimaksie perspektywami. Aby przeprowadzić takie badania, zaimplementowano dopasowany do problematyki algorytm genetyczny i uruchomiono kilka jego sesji na przestrzeni paru miesięcy. 

Praca podzielona jest na 4 główne rozdziały:

\begin{enumerate}
    \item \nameref{rozdzial1}. Rozdział ten zawiera wprowadzenie do gry w warcaby, omówienie reguł gry oraz definicję wariantu angielskiego tej gry. Zawarto również krótką informację na temat dotychczasowych badań warcabów w~tym wariancie.
    \item \nameref{rozdzial2}. Określone tu zostały wykorzystane w pracy algorytmy, jak i cele ich użycia. Omawiane są po kolei algorytm Minimax, funkcja oceny heurystycznej i algorytm genetyczny. Rozdział wzbogacony jest również o rysunki ilustrujące zachowania algorytmów.
    \item \nameref{rozdzial3}. Rozdział poświęcony jest wdrożeniu omawianych wcześniej algorytmów w rzeczywisty projekt napisany w konkretnym języku programowania. Podane są ogólne zadania i funkcjonalności najważniejszych części projektu. Na końcu rozdziału zapisano instrukcję obsługi programów wykonawczych.
    \item \nameref{rozdzial4}. W ostatnim rozdziale znajdują się opisy przeprowadzonych eksperymentów, wyniki tych eksperymentów oraz wnioski. Dodatkowo w osobnym podrozdziale zawarto pomysły na rozszerzenie projektu i badań.
\end{enumerate}


