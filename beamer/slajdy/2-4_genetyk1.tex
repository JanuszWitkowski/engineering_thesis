\begin{frame}{Algorytm genetyczny}

	\begin{itemize}
		\myitem Celem pracy jest wyznaczenie jak najlepszego ciągu wag do funkcji oceny heurystycznej.
		\myitem Ludzka intuicja może w tym zadaniu zawieść.
		\myitem Pomysł: zastosowanie algorytmu genetycznego.
	\end{itemize}
	
	\hspace{1cm}

	\textbf{Algorytm genetyczny} symuluje dobór naturalny w przyrodzie. Na populacji osobników (zbiorze rozwiązań problemu) wykonuje się operacje:
	\begin{enumerate}
		\item Ewaluacja osobników
		\item Selekcja zbioru rodziców
		\item Krzyżowanie 
		\item Losowe mutacje
		\item Tworzenie nowej populacji
	\end{enumerate}
	
	% \begin{columns}	
			
	% 	\begin{column}{.45\hsize}
			
	% 		\begin{tikzpicture}[remember picture, overlay]
	% 			\node [shift={(2.2 cm,-4.5cm)}]  at (current page.north west)
	% 			{
	% 				\tikzstyle{place}=[circle,draw=blue!50,fill=blue!20,thick,inner sep=0pt,minimum size=4mm]
	% 				\tikzstyle{red place}=[circle,draw=red!50,fill=yellow,thick,inner sep=0pt,minimum size=4mm]
					
	% 				{\scalefont{0.6}
						
	% 					\begin{tikzpicture}
							
	% 						\only<1>{
	% 							\node at (-1,1) [place]  (1) {$0.45$};
	% 							\node at (-1,3) [place]  (2) {$0.76$};
	% 							\node at ( 0,2) [place]  (3) {$0.49$};
	% 							\node at ( 0.2,0) [place]  (4) {$0.78$};
	% 							\node at (1.5,2.5) [place]  (6) {$0.67$};
	% 							\node at ( 2,-0.2) [place] (7) {$0.86$};
	% 							\node at (2.5,1.7) [place]  (8) {$0.22$};}
							
	% 						\only<1,2>{
	% 							\node at ( 1,1) [ red place] (5) {$0.01$};}
							
	% 						\only<2>{
	% 							\node at (-1,1) [red place] (1) {$0.01$};
	% 							\node at (-1,3) [red place]  (2) {$0.01$};
	% 							\node at ( 0,2) [red place](3) {$0.01$};
	% 							\node at ( 0.2,0) [red place]  (4) {$0.01$};
	% 							\node at (1.5,2.5) [red place] (6) {$0.01$};
	% 							\node at ( 2,-0.2) [red place](7) {$0.01$};
	% 							\node at (2.5,1.7) [red place]  (8) {$0.01$};}
							
	% 						\draw [->,-latex] (1) to (2);
	% 						\draw [<->,-latex] (2) to (3);
	% 						\draw [<->,-latex]  (3) to (6);
	% 						\draw [<->,-latex]  (3) to (5);
	% 						\draw [<->,-latex]  (5) to (6);
	% 						\draw [->,-latex]  (6) to (8);
	% 						\draw [->,-latex]  (8) to (7);
	% 						\draw [->,-latex]  (7) to (5);
	% 						\draw [->,-latex]  (5) to (4);
	% 						\draw [<->,-latex]  (1) to (4);
							
	% 					\end{tikzpicture}
	% 				}
	% 			};
	% 		\end{tikzpicture}
			
	% 	\end{column}		
		
	% 	\begin{column}{.75\hsize}		
				
	% 			\textbf{Model}
	% 			\begin{enumerate}
	% 				\myitem model sieci: graf skierowany $\mathcal{G}=(V,E)$
	% 				\myitem każda stacja $v_i \in V$ losuje $x_i\in[0,1]$
	% 			\end{enumerate}
				
	% 			\vspace{0.4cm}
				
	% 			\textbf{Cel}
	% 			\begin{enumerate}
	% 				\myitem propagacja statystyk pozycyjnych, 
	% 				\newline np. $\min(x_1,\ldots, x_n)$
	% 			\end{enumerate}
				
				
	% 			\vspace{0.4cm}
				
	% 			\textbf{Dlaczego statystyki pozycyjne?}
	% 			\begin{enumerate}
	% 				\myitem zastosowanie do rozproszonego obliczania dowolnej funkcji \textit{separowalnej}\\
	% 			\end{enumerate}
						
	% 	\end{column}
			
	% \end{columns}
	
\end{frame}
