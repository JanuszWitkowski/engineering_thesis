\begin{frame}{Wyniki: różne głębokości}

    \textbf{Cel eksperymentu:} Sprawdzić czy puszczenie dwóch sesji algorytmu genetycznego z różnymi głębokościami przeszukiwań (odpowiednio 4 i 5) dają różne rezultaty.

    \textbf{Wyniki:}
    \textit{TBA}.
    % \begin{itemize}
    %     \myitem
    % \end{itemize}

    % \textbf{Zadanie:} Napisz program który stwierdzi czy dana liczba naturalna jest pierwsza.

    % \vspace*{0.5cm}

    % \begin{tcolorbox}[title={C, 61 Bajtów {\color{blue} \hyperlink{frame:przypisy}{(10)}}}]
    %     % r;main(i,j){r=(--i>1);for(j=i-1;j>1;)r*=!!(i\%j--);return r;}
    %     {\includegraphics[width=11cm]{figures/primes_c.png}}
    % \end{tcolorbox}

    % \begin{tcolorbox}[title={Perl, 25 Bajtów {\color{blue} \hyperlink{frame:przypisy}{(11)}}}]
    %     % \$\_=2==grep\$'\%\$\_<1,//..\$\_
    %     {\includegraphics[width=11cm]{figures/primes_perl.png}}
    % \end{tcolorbox}

    % \begin{tcolorbox}[title={APL, 13 Bajtów {\color{blue} \hyperlink{frame:przypisy}{(12)}}}]
    %     % 2=+/0=x|⍨⍳x←⎕
    %     {\includegraphics[width=11cm]{figures/primes_apl.png}}
    % \end{tcolorbox}

\end{frame}
