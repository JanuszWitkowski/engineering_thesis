\begin{frame}{Wyniki: wagi parametrów}

    \textbf{Cel eksperymentu:} Poznać względne wagi wszystkich parametrów; sprawdzić które parametry nie mają znaczenia w rozgrywce (waga bliska zeru).

    \textbf{Wyniki:}
    \begin{itemize}
        \myitem Najważniejszymi parametrami w rozgrywce okazały się \textit{X, Y, Z (omówienie)}.
        \myitem Najgorzej punktowane parametry to \textit{A, B, C (omówienie)}.
        \myitem Najmniejszy wpływ na grę mają parametry \textit{U, V, W (omówienie)}. Będzie można odpuścić te parametry w następnych eksperymentach.
    \end{itemize}
    % Czym jest Code Golf?
    % \begin{itemize}
    %     \myitem Programistyczny ,,sport''
    %     \myitem Zapisanie programu o danej funkcjonalności w jak najmniejszej liczbie znaków
    %     \myitem Czytelność kodu, złożoność obliczeniowa ani złożoność pamięciowa nie grają roli
    % \end{itemize}

    % \vspace*{0.3cm}
    % % Istnieją strony i konkursy poświęcone Code Golfingowi i/lub Obfuscated Codingowi:

    % \begin{columns}
    %     \begin{column}{.25\hsize}
    %         {\hspace*{0.7cm}\includegraphics[height=1.4cm]{figures/codegolf_logo.png}}
    %     \end{column}
    %     \begin{column}{.75\hsize}
    %         Code Golf Stack Exchange
    %     \end{column}
    % \end{columns}

    % \begin{columns}
    %     \begin{column}{.25\hsize}
    %         {\hspace*{0.7cm}\includegraphics[height=1.4cm]{figures/perl_logo.png}}
    %     \end{column}
    %     \begin{column}{.75\hsize}
    %         Perl Golf
    %     \end{column}
    % \end{columns}

    % \begin{columns}
    %     \begin{column}{.25\hsize}
    %         {\hspace*{0.7cm}\includegraphics[height=1.4cm]{figures/ioccc_logo.jpeg}}
    %     \end{column}
    %     \begin{column}{.75\hsize}
    %         International Obfuscated C Code Contest
    %     \end{column}
    % \end{columns}

    % \begin{tcolorbox}[title={\centering Sprawdzenie czy napisy są swoimi anagramami, C}]
    %     t[256],i;main(c){for(;c+3;)(i=getchar())>10?t[i]+=c:(c-=2);for(i=257;--i\&\&!t[i-1];);puts(i?"false":"true");}
    % \end{tcolorbox}
\end{frame}
